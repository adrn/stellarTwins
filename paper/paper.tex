% This file is part of the stellarTwins project.
% Copyright 2016 the authors.
\documentclass[11pt,letterpaper]{article}

\newcommand{\acronym}[1]{{\small{#1}}}
\newcommand{\project}[1]{\textsl{#1}}
\newcommand{\Gaia}{\project{Gaia}}
\newcommand{\GALEX}{\project{\acronym{GALEX}}}
\newcommand{\APASS}{\project{\acronym{APASS}}}
\newcommand{\RAVEon}{\project{\acronym{RAVE}-on}}

\linespread{1.09}
\setlength{\parindent}{1.2\baselineskip}
\sloppy\sloppypar\raggedbottom\frenchspacing
\begin{document}

\section*{\raggedright%
A photometric method for obtaining spectroscopic parameters of stars,
with no use of physical models}

\noindent
Lauren Anderson (Flatiron), David W. Hogg (Flatiron), others

\paragraph{Abstract:}
% Context
Stellar photospheres, to first order, have their spectral properties
set by the local quantities (spectroscopic parameters) effective
temperature, surface gravity, and metallicity (chemical composition);
As their name suggests, spectroscopic parameters can be determined
with (medium to high resolution) spectral data and photosphere models.
Global stellar structure, including radii, masses, and bolometric
luminosities cannot be predicted theoretically without a physical
model of stellar structure or stellar evolution.
The \Gaia\ Mission has already delivered parallaxes---and therefore
luminosity information---for millions of stars; \GALEX\ and
\APASS\ deliver multi-band photometry; for a subset, \RAVEon\ delivers
spectroscopic parameters.
% Aims
In this project, we infer spectroscopic parameters for stars for which
only photometry and astrometry is available.
We make no use of stellar structure or evolution models.
% Methods
After showing empirically that stars with similar luminosities and
colors have similar spectroscopic parameters, we train a very flexible
regression model of the relationship between luminosity, color, and
the spectroscopic parameters.
The model is a modification of a k-nearest-neighbors method.
We use this model to paint spectroscopic parameters onto stars that have
not been observed spectroscopically.
% Results
We find XXX and YYY. Things degrade in ZZZ way when WWW.

\section{Introduction}

Label transfer, The Cannon, etc.

\section{Method}

magSN = parallaxSN = 16 (including systematic parallax error in tgas)

minimum galactic latitude = 5 degrees

no NaNs in apass photometry, apass errors > 0

matched in Apass, wise, 2mass, rave

check that matching looks good with extreme colors

calculate distance modulus from sampling of gaia distances

correct for dust using Bayestar (Schlafly+Finkbeiner); minimal dust to matched stars

use absolute V band magnitude, B-V, g-r r-i colors

build KDTree of matched stars in Mv/color space

grab 2048 nearest neighbors

calculate chisq like distance metric between each chosen star and its neighbors

generate posterior belief of a chosen stars spectroscopic characteristics (ie Teff, logg, FeH)

given the spectroscopic characteristics (and uncertainties) of its nearest neighbors in Mv/color space

\section{Discussion}

Apass HR Diagram looks very different from Rave HR Diagram, MS stars are incorrectly being labeled as giants

\paragraph{Acknowledgements:}
This project was developed in part at the 2016 NYC Gaia Sprint, hosted
by the Center for Computational Astrophysics at the Simons Foundation
in New York City.

This work has made use of data from the European Space Agency (ESA)
mission \Gaia\ (http://www.cosmos.esa.int/gaia), processed by the \Gaia\ %
Data Processing and Analysis Consortium (\acronym{DPAC},
http://www.cosmos.esa.int/web/gaia/dpac/consortium). Funding for the
\acronym{DPAC} has been provided by national institutions, in particular the
institutions participating in the Gaia Multilateral Agreement.

\end{document}
