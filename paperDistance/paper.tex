% This project is part of the stellarTwins project.
% Copyright 2017 the authors.

% # To-do list
% - zeroth draft the introduction
% - zeroth draft the method section
% - zeroth draft the assumptions list
% - reformat single-column

\documentclass[iop]{emulateapj}
\newcommand{\myemail}{landerson@simonsfoundation.org}

\slugcomment{Stars are kinda cool again.}

\shorttitle{Measuring Mo' Betta Distances}
\shortauthors{Anderson et al}

\newcommand{\acronym}[1]{{\small{#1}}}
\newcommand{\project}[1]{\textsl{#1}}
\newcommand{\tgas}{\project{\acronym{TGAS}}}
\newcommand{\tmass}{\project{\acronym{2MASS}}}
\newcommand{\gaia}{\project{Gaia}}
\newcommand{\xd}{\acronym{XD}}

\begin{document}\sloppy\sloppypar\raggedbottom\frenchspacing

\title{Mo' Betta Gaia Distance Measurements, \\
    Using Gaia + APASS photometry}

\author{Lauren Anderson\altaffilmark{1},
        David W. Hogg\altaffilmark{1, 2, 3, 4},
        Boris Leistedt\altaffilmark{2},
        Adrian Price-Whelan\altaffilmark{5}}
\altaffiltext{1}{Center for Computational Astrophysics, Flatiron Institute, Simons Foundation, New York City}
\altaffiltext{2}{Center for Cosmology and Particle Physics, Department of Physics, New York University}
\altaffiltext{3}{Center for Data Science, New York University}
\altaffiltext{4}{Max-Planck-Institut f\"ur Astronomie, Heidelberg}
\altaffiltext{5}{Princeton University Observatory, Princeton University}

\begin{abstract}
The \gaia\ \tgas\ \project{Catalog} contains more than 2 million parallax measurements.
Conversion of a noisy parallax measurement into a posterior belief over distances requires inference with a prior.
Usually this prior represents some kind of belief about the Milky Way.
However, there is multi-band photometry for the \tgas\ stars from imaging surveys;
this imaging is incredibly informative about stellar distances.
Here we use color information on \tgas\ stars from \tmass\ to build a noise-deconvolved empirical prior distribution of stars in color--magnitude space.
This data-driven model contains no knowledge of the physics of stellar interiors or photospheres, nor of the Milky Way, but rather derives its precision from its generative model of noisy parallax measurements and an assumption of stationarity.
We use the Extreme Deconvolution (\xd) algorithm, which is an Empirical Bayes approximation to a full hierarchical model of the true parallax and photometry of every star.
The algorithm is run not in absolute-magnitude space but in a transformed space where the measurement uncertainty is closer to Gaussian.
The \xd-optimized prior is used to perform parallax and distance inferences for every star, yielding a precise stellar distance estimate and uncertainty (and full posterior) for every star.
Our posterior parallax estimates are more precise than the Gaia catalog outputs by a factor of [XXX] for the median \tgas\ star; and more precise than previous examples of Bayesian distance estimates by a factor of [YYY].
The precision can be attributed to the statistics concept of shrinkage.
We independently validate our distances by looking at members of Milky Way star clusters; for example, M67 is not visible at all in the \tgas\ parallax estimates, but appears clearly in our posterior parallax estimates.
All our results, including a posterior parallax and distance sampling for [ZZZ] \tgas\ stars, are available in companion electronic tables.
\end{abstract}

\keywords{
  catalogs
  ---
  Hertzsprung--Russell and C--M diagrams
  ---
  methods: statistical
  ---
  parallaxes
  ---
  stars: distances
  ---
  stars: statistics
}

\section{Introduction}

The \gaia\ Mission will deliver more than a billion stellar distances.
Only a small fraction (but large number) of these distance
measurements will be purely astrometric:
\gaia\ uses astrometric parallax to determine the distances of the closer
stars, and calibrate spectrophotometric models.
These spectrophotometric models, in turn, along with \gaia's on-board
low-resolution $B_p\,R_p$ spectrophotometry, are used to provide
distance estimates to more distant stars.
The full stack required for these distance inferences is complex.
It involves modeling not just stars, but also the dust in the Milky Way,
and the response of the telescope itself.

With projects like \project{The Cannon} (DWH CITE) and \project{Avast}
(DWH CITE), we are exploring the extent to which our predictive models of
stars could be purely data-driven or statistical.
That is, under what circumstances could the data themselves deliver
more precise or more accurate models than any theoretical or
physics-based model?

[DWH: NEEDS WORK] Given the likelihood measurements of the parallax to local stars from Gaia, and using prior information for the distances to these stars built from the data themselves, we make more precise measurements of the distances to local stars in Gaia than the parallax measurements alone. The prior, which is very similar to a color-magnitude diagram of stars, is informative because stars fall on a locus within this space. By deconvolving the set of observed points with their uncertainties we tighten this locus which therefore informs the individual measurements that they are drawn from a tighter distribution. We build this prior using Extreme Deconvolution. Extreme Deconvolution is a gaussian mixture model of data, specifically data that have heteroscedastic uncertainties, a quality that gaussin mixture models alone cannot take into account. This prior does not include any assumptions about the physical distribution of stars within the galaxy, nor does it include any stellar models.

\section{Data}

We use stars cross matched in the TGAS and APASS set of observations. We require there are no NaNs in any APASS photometric band, as well as every band having positive errors. We check that the match between APASS and TGAS is good using extreme APASS-WISE colors. We correct for dust using 3D Bayestar (Schlafly+Finkbeiner) with an estimate of the distance from sampling of posterior Gaia distances from Adrain. There is minimal dust to the matched stars. To build our model here we use the Gaia G band magnitude, parallax and APASS B-V color.

\section{Methods}

We build a prior for the distance measurement of Gaia stars using the Gaia measured parallaxes plus APASS photometry to build a color-magnitude-like diagram that serves as the prior for distance measurements of Gaia stars. Instead of making assumptions about the physical distributions of stars in the galaxy, or using stellar models for the modeled HR diagram of stars, we build a prior from the data themselves.


magSN = parallaxSN = 16 (excluding systematic parallax error in tgas)
train gaussian means, mus, Vs on progressively lower SN dataset

\begin{equation}
\label{eq:bayes}
P(\theta|\textbf{E}, \textbf{I}) = P(\theta ) \frac{P(\textbf{E} |\theta,\textbf{I})}{P(\textbf{E})},
\end{equation}


where $\theta$ is the model parameters that generate the evidence $\textbf{E}$, and P($\theta$) is the prior knowledge of the model, and $P(\textbf{E} |\theta,\textbf{I})$ is the likelihood of the evidence, given the model and some other prior assumptions. Our evidence is the parallaxes and G band magnitudes from Gaia, and the B and V band magnitudes from APASS. We build the prior from this evidence using extreme deconvolution


Crossvalidation

\subsection{Formalism} \label{bozomath}

\section{Results}

\section{Discussion}
Apass HR Diagram looks very different from Rave HR Diagram, MS stars are incorrectly being labeled as giants

Long discussion of dust

Number of gaussians in XD, and the w parameter



\acknowledgments It is a pleasure to thank Andy Casey (Monash) and the
attendees at the Stars Group Meeting at the CCA for comments and
input.

This project was developed in part at the 2016 \acronym{NYC} Gaia Sprint, hosted
by the Center for Computational Astrophysics at the Simons Foundation
in New York City.

This work has made use of data from the European Space Agency (\acronym{ESA})
mission Gaia (http://www.cosmos.esa.int/gaia), processed by the Gaia
Data Processing and Analysis Consortium (\acronym{DPAC},
http://www.cosmos.esa.int/web/gaia/dpac/consortium). Funding for the
\acronym{DPAC} has been provided by national institutions, in particular the
institutions participating in the Gaia Multilateral Agreement.

[IS THERE A \tmass\ ACKNOWLEDGEMENT?]

This project was partially supported by [DWH GIVE GRANT NUMBERS]. It
made use of the \acronym{NASA} Astrophysics Data System.

\appendix

\section{Appendix material}


\begin{thebibliography}{}
\end{thebibliography}

\clearpage

\end{document}



%%
%% End of file `sample.tex'.
